{\rtf1\ansi\ansicpg936\cocoartf1404\cocoasubrtf460
{\fonttbl\f0\fswiss\fcharset0 Helvetica;}
{\colortbl;\red255\green255\blue255;}
\paperw11900\paperh16840\margl1440\margr1440\vieww10800\viewh8400\viewkind0
\pard\tx720\tx1440\tx2160\tx2880\tx3600\tx4320\tx5040\tx5760\tx6480\tx7200\tx7920\tx8640\pardirnatural\partightenfactor0

\f0\fs24 \cf0 \\documentclass[letterpaper,12pt]\{article\}\
\\usepackage\{array\}\
\\usepackage\{threeparttable\}\
\\usepackage\{geometry\}\
\\geometry\{letterpaper,tmargin=1in,bmargin=1in,lmargin=1.25in,rmargin=1.25in\}\
\\usepackage\{fancyhdr,lastpage\}\
\\pagestyle\{fancy\}\
\\lhead\{\}\
\\chead\{\}\
\\rhead\{\}\
\\lfoot\{\}\
\\cfoot\{\}\
\\rfoot\{\\footnotesize\\textsl\{Page \\thepage\\ of \\pageref\{LastPage\}\}\}\
\\renewcommand\\headrulewidth\{0pt\}\
\\renewcommand\\footrulewidth\{0pt\}\
\\usepackage[format=hang,font=normalsize,labelfont=bf]\{caption\}\
\\usepackage\{listings\}\
\\lstset\{frame=single,\
  language=Python,\
  showstringspaces=false,\
  columns=flexible,\
  basicstyle=\{\\small\\ttfamily\},\
  numbers=none,\
  breaklines=true,\
  breakatwhitespace=true\
  tabsize=3\
\}\
\\usepackage\{amsmath\}\
\\usepackage\{amssymb\}\
\\usepackage\{amsthm\}\
\\usepackage\{harvard\}\
\\usepackage\{setspace\}\
\\usepackage\{float,color\}\
\\usepackage[pdftex]\{graphicx\}\
\\usepackage\{hyperref\}\
\\hypersetup\{colorlinks,linkcolor=red,urlcolor=blue\}\
\\theoremstyle\{definition\}\
\\newtheorem\{theorem\}\{Theorem\}\
\\newtheorem\{acknowledgement\}[theorem]\{Acknowledgement\}\
\\newtheorem\{algorithm\}[theorem]\{Algorithm\}\
\\newtheorem\{axiom\}[theorem]\{Axiom\}\
\\newtheorem\{case\}[theorem]\{Case\}\
\\newtheorem\{claim\}[theorem]\{Claim\}\
\\newtheorem\{conclusion\}[theorem]\{Conclusion\}\
\\newtheorem\{condition\}[theorem]\{Condition\}\
\\newtheorem\{conjecture\}[theorem]\{Conjecture\}\
\\newtheorem\{corollary\}[theorem]\{Corollary\}\
\\newtheorem\{criterion\}[theorem]\{Criterion\}\
\\newtheorem\{definition\}[theorem]\{Definition\}\
\\newtheorem\{derivation\}\{Derivation\} % Number derivations on their own\
\\newtheorem\{example\}[theorem]\{Example\}\
\\newtheorem\{exercise\}[theorem]\{Exercise\}\
\\newtheorem\{lemma\}[theorem]\{Lemma\}\
\\newtheorem\{notation\}[theorem]\{Notation\}\
\\newtheorem\{problem\}[theorem]\{Problem\}\
\\newtheorem\{proposition\}\{Proposition\} % Number propositions on their own\
\\newtheorem\{remark\}[theorem]\{Remark\}\
\\newtheorem\{solution\}[theorem]\{Solution\}\
\\newtheorem\{summary\}[theorem]\{Summary\}\
%\\numberwithin\{equation\}\{section\}\
\\bibliographystyle\{aer\}\
\\newcommand\\ve\{\\varepsilon\}\
\\newcommand\\boldline\{\\arrayrulewidth\{1pt\}\\hline\}\
\
\
\\begin\{document\}\
\
\\begin\{flushleft\}\
  \\textbf\{\\large\{Problem Set \\#4\}\} \\\\\
  MACS 30000, Dr. Evans \\\\\
  Xingyun Wu\
\\end\{flushleft\}\
\
\\vspace\{5mm\}\
\
\\noindent\\textbf\{Problem 1: Some income data, lognormal distribution, and SMM.\}\
\
\\vspace\{2mm\}\
\
\\textbf\{Part (a).\} The histogram is shown in Figure 1.\
\\begin\{figure\}[htb]\\centering\\captionsetup\{width=4.0in\}\
  \\caption\{\\textbf\{Histogram of Percentages of the MACSS Graduates\}\}\\label\{Fig1a\}\
  \\fbox\{\\resizebox\{4.0in\}\{3.0in\}\{\\includegraphics\{Fig_1a.png\}\}\}\
\\end\{figure\}\
\
\\textbf\{Part (b).\} Testing my function by inputting the given matrix and parameter values:\
\\begin\{bmatrix\}\
0.0019079 &  0.00123533\\\\\
0.00217547  & 0.0019646\\\\\
\\end\{bmatrix\}\\\\\
\
\\vspace\{2mm\}\
\
\\textbf\{Part (c).\} Figure 2 shows the PDF of lognormal distribution by one-step SMM estimation against the histogram from part(a).\\\\\
The estimates are: mu = 11.3306370447, sigma = 0.209229359523\\\\\
The value of SMM criterion function is: 5.80588645143e-14\\\\\
The data moments are: mean = 85276.8236063, standard deviation = 17992.542128\\\\\
The model moments are: mean = 85276.8115546, standard deviation = 17992.5386167\
\
\\vspace\{1mm\}\
\
Using mu=9.0 and sigma=0.3 as the initial guess, the SMM estimation provide estimated mu and sigma as above. The model moments are very close to the data moments, which means the SMM estimation performs well.\
\
\\begin\{figure\}[htb]\\centering\\captionsetup\{width=4.0in\}\
  \\caption\{\\textbf\{Lognormal PDF of one-step estimation\}\}\\label\{Fig1c\}\
  \\fbox\{\\resizebox\{4.0in\}\{3.0in\}\{\\includegraphics\{Fig_1c.png\}\}\}\
\\end\{figure\}\
\
\\vspace\{2mm\}\
\
\\textbf\{Part (d).\} Figure 3 shows the PDF of lognormal distribution by two-step SMM estimation against the histogram from part(a) and the estimated PDF from part(c).\
\\\\\
The estimates are: mu = 11.3306371028, sigma = 0.209229396992\\\\\
The value of the two-step SMM criterion function at the estimated parameter values is: 0.000903894673886 ( greater than 5.80588645143e-14)\\\\\
The data moments are: mean = 85276.8236063, standard deviation = 17992.542128\\\\\
The model moments of one-step estimation are: mean = 85276.8115546, standard deviation = 17992.5386167\\\\\
The model moments of two-step estimation are: mean = 85276.8171949, standard deviation = 17992.5430962\
\
\\vspace\{1mm\}\
\
Model moments of both one-step estimation and two-step estimation are very close to the data moments. The model moments of two-step estimation are closer to the data moments. But the value of the two-step SMM criterion function are greater than the value of the one-step SMM criterion function, which means the two-step estimation is actually producing more errors. So with comparison to data moments, the two-step estimation performs better than the one-step estimation, but it does not guarantee less estimation errors.\
\
\\vspace\{5mm\}\
\\noindent\\textbf\{Please see the next page for Figure 3.\}\
\
\\begin\{figure\}[htb]\\centering\\captionsetup\{width=4.0in\}\
  \\caption\{\\textbf\{Lognormal PDF of two-step estimation\}\}\\label\{Fig1c\}\
  \\fbox\{\\resizebox\{4.0in\}\{3.0in\}\{\\includegraphics\{Fig_1d.png\}\}\}\
\\end\{figure\}\
\
\\end\{document\}\
\
}