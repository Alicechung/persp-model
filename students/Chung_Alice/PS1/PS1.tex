\documentclass[letterpaper,12pt]{article}
\usepackage{float}
\usepackage{array}
\usepackage{threeparttable}
\usepackage{geometry}
\geometry{letterpaper,tmargin=1in,bmargin=1in,lmargin=1.25in,rmargin=1.25in}
\usepackage{fancyhdr,lastpage}
\usepackage{fouriernc,amsmath}
\pagestyle{fancy}
\lhead{}
\chead{}
\rhead{}
\lfoot{}
\cfoot{}
\rfoot{\footnotesize\textsl{Page \thepage\ of \pageref{LastPage}}}
\renewcommand\headrulewidth{0pt}
\renewcommand\footrulewidth{0pt}
\usepackage[format=hang,font=normalsize,labelfont=bf]{caption}
\usepackage{listings}
\lstset{frame=single,
  language=Python,
  showstringspaces=false,
  columns=flexible,
  basicstyle={\small\ttfamily},
  numbers=none,
  breaklines=true,
  breakatwhitespace=true
  tabsize=3
}
\usepackage{amsmath}
\usepackage{amssymb}
\usepackage{amsthm}
\usepackage{harvard}
\usepackage{setspace}
\usepackage{float,color}
\usepackage[pdftex]{graphicx}
\usepackage{hyperref}
\hypersetup{colorlinks,linkcolor=red,urlcolor=blue}
\theoremstyle{definition}
\newtheorem{theorem}{Theorem}
\newtheorem{acknowledgement}[theorem]{Acknowledgement}
\newtheorem{algorithm}[theorem]{Algorithm}
\newtheorem{axiom}[theorem]{Axiom}
\newtheorem{case}[theorem]{Case}
\newtheorem{claim}[theorem]{Claim}
\newtheorem{conclusion}[theorem]{Conclusion}
\newtheorem{condition}[theorem]{Condition}
\newtheorem{conjecture}[theorem]{Conjecture}
\newtheorem{corollary}[theorem]{Corollary}
\newtheorem{criterion}[theorem]{Criterion}
\newtheorem{definition}[theorem]{Definition}
\newtheorem{derivation}{Derivation} % Number derivations on their own
\newtheorem{example}[theorem]{Example}
\newtheorem{exercise}[theorem]{Exercise}
\newtheorem{lemma}[theorem]{Lemma}
\newtheorem{notation}[theorem]{Notation}
\newtheorem{problem}[theorem]{Problem}
\newtheorem{proposition}{Proposition} % Number propositions on their own
\newtheorem{remark}[theorem]{Remark}
\newtheorem{solution}[theorem]{Solution}
\newtheorem{summary}[theorem]{Summary}
%\numberwithin{equation}{section}
\bibliographystyle{aer}
\newcommand\ve{\varepsilon}
\newcommand\boldline{\arrayrulewidth{1pt}\hline}


\begin{document}

\begin{flushleft}
  \textbf{\large{Problem Set \#1}} \\
  MACS 30100, Dr. Evans \\
  Alice Mee Seon Chung
\end{flushleft}

\vspace{2mm}

\noindent\textbf{Problem 1: Classify a model from a journal}
 \\
 
\textbf{Part(b)} Kovak, Brian K..2013.{"}Regional Effects of Trade Reform: What Is the Correct Measure of Liberalization?{"}\textit {American Economic Review},103(5): 1960-76.
  
\par\textbf{Part(c-g)}
 In the paper { "Regional Effects of Trade Reform: What is the Correct Measure of Liberalization?"} Brian K. Kovak measures the local effect of liberalization using a weighted average of changes in trade policy. The model showed that liberalization in a certain industry in a situation that the liberalization has larger effect on the prices and the industry have larger portion of the local employment and labor demand is elastic.
 \par
He uses specific-factor model of regional economics to estimate prices changes{'} effects on regional wages. With this model, trade liberalization{'}s effect on a region{'}s wages is determined by a weighted average of liberalization-induced price changes. He called this weighted average as {"regional-level tariff change(RTC)"}. He examined the effect of tariff changes on regional wages from specific-factor model. The estimating equation is
\begin{equation} \label{eu_eqn}
\textit{d}\ln\left ( w_{r} \right ) = \zeta _{0} + \zeta _{1}RTC_{r} + \epsilon _{r}
\end{equation}
 \par
 In this model, exogenous variable which is input is region-level tariff changes (RTC) in r that stands for a specific country with many regions. The regional effect of liberalization on real wages between 1991 and 2000 is represented as $\zeta_{r}$. He expected that 0<$\zeta_{r}$<1. The error term $\epsilon_{r}$ means any unobserved factors of wages changes that are not associated with liberalization. With this model, we can figure out the output, the endogenous variable, $\textit{d}\ln( w_{r})$ which means the regional wage changes. 

 \par 
I classify this model as static, linear and stochastic model. First, I think the model is static because the model does not contain any variables related with the time. To be specific, the concept that he wants to measure, regional wage changes, interacts with the local labor market and the elasticity of the labor demand.  Second, this model is linear model. It is clear that its equation takes only one independent variable and it resembles the form of linear regression model. The equation explains in the way of combination of independent variable and regional-wage tariff changes which roles as a coefficient of the equation. Third, I believe the model is stochastic model. The output of the model is determined by the parameters and the assumptions. In the equation, the RTC can interact and change through the specific conditions and it does consider the error term which means the randomness inside. 

\par
A variable of feature that might be valuable is the local government{'}s subsidy. In the paper, the model considers homogenous labor, perfect competition, mobility between industries and fully employed. However, these features are in the side of employees, so I think the model has to consider the feature in the side of the employer. If the government subsidized the company, the wage-tariff relationship driven by the model is not fully explained by the equation itself. The change will occur, but the scale of the change may not be significant if the subsidy is considered in the equation.  

 \newpage
 
 \noindent\textbf{Problem 2: Make your own model}

\textbf{Part(a)}            
\begin{multline}
Y_{genre_{i,j}} = \beta _{0}+\beta _{1}X_{stress\, level}+\beta _{2}X_{family\, history}+\beta _{3}X_{alcohol}+\beta _{4}X_{drug} \\
+\beta _{5}X_{work\, out}+\beta _{6}X_{medical\, examination}+\beta _{7}X_{mental\, care}\\
+\gamma _{average\, life\, span}+\epsilon_{genre_{i,j}}
\end{multline}

\textbf{Part(b-e)}
The model predicted the life span of a particular musician \textit{j} in certain genre i in years. The output $Y_{genre_{i,j}}$ indicates the predicted life span that in a certain genre \textit{i} and for certain musician \textit{j}. Genre \textit{i} will be one of  the genre set (e.g rock, pop, hiphop/rap, country, R\&B). $X_{stress\, level}$ present the musician{'}s anxiety as stress level, $X_{family\, history}$ considers musician{'}s own health status through the family history. $X_{alcohol}$ and $X_{drug}$ indicate the factors that harm health status and $X_{work\, out}$,$X_{medical\, examination}$,$X_{mental\, care}$ consider that help to extend the lifespan in positive way. $\gamma _{average\, life\, span }$ denotes the average life span of people and the last factor $\epsilon_{genre_{i,j}}$ will consider the person{'}s any own randomness events in the equation. The all variables in this equation will be the key factors that influence the outcome. 
\par 
The all variables in this equation are important but the variable of $X_{stress\, level}$ and $X_{mental\, care}$ would be the key factors that influence the outcome of the model. For average people, the stress level is the important factor to live a healthy lifestyle. I consider the popular musicians must live high stress lifestyle than average people. With this lifestyle, the musicians would try many things to lessen stress. However, the way to lessen stress could be done in two different ways, positive and negative way. I think the variable $X_{mental\, care}$ would measure the musicians deal with stress in positive way or negative way. 

\par
I decide those factors on the assumption that the popular musicians are average people so the model should consider the common factors regarding health status. Besides those common factors like the life span of average people and family history, I consider the factors that influence health status into negative and positive way. The popular musicians tend to expose to lots of drugs, alcohols and have high stress level which will work in negative way to health status. I also considers the work out frequencies, medical examination and mental care since I assumed that controlling their high stress level, high pressure and depressive disorder is significant when considering the cases of popular musicians died young.\\
\par\textbf{Part(f-g)}
A preliminary test of these factors would be examined by collecting small sample of died popular musician data set in every genre in the genre set. The reason that I choose to test the model to died musicians is because the model measures the predicted life span of musician, but it is impossible to apply or do the test the model to live musicians. For validity and reliability, the size of sample should be more than the size of 30. I will collect the sample data set using Wikipedia. By using regression model, I will compare between actual life span and predicted life span through compare every R-squared and MSE to determine the factors are significant and validate in real life. 




\end{document}
