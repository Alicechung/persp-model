\documentclass[letterpaper,12pt]{article}
\usepackage{array}
\usepackage{threeparttable}
\usepackage{geometry}
\geometry{letterpaper,tmargin=1in,bmargin=1in,lmargin=1.25in,rmargin=1.25in}
\usepackage{fancyhdr,lastpage}
\pagestyle{fancy}
\lhead{}
\chead{}
\rhead{}
\lfoot{}
\cfoot{}
\rfoot{\footnotesize\textsl{Page \thepage\ of \pageref{LastPage}}}
\renewcommand\headrulewidth{0pt}
\renewcommand\footrulewidth{0pt}
\usepackage[format=hang,font=normalsize,labelfont=bf]{caption}
\usepackage{listings}
\lstset{frame=single,
  language=Python,
  showstringspaces=false,
  columns=flexible,
  basicstyle={\small\ttfamily},
  numbers=none,
  breaklines=true,
  breakatwhitespace=true
  tabsize=3
}
\usepackage{amsmath}
\usepackage{amssymb}
\usepackage{amsthm}
\usepackage{harvard}
\usepackage{setspace}
\usepackage{float,color}
\usepackage[pdftex]{graphicx}
\usepackage{hyperref}
\hypersetup{colorlinks,linkcolor=red,urlcolor=blue}
\theoremstyle{definition}
\newtheorem{theorem}{Theorem}
\newtheorem{acknowledgement}[theorem]{Acknowledgement}
\newtheorem{algorithm}[theorem]{Algorithm}
\newtheorem{axiom}[theorem]{Axiom}
\newtheorem{case}[theorem]{Case}
\newtheorem{claim}[theorem]{Claim}
\newtheorem{conclusion}[theorem]{Conclusion}
\newtheorem{condition}[theorem]{Condition}
\newtheorem{conjecture}[theorem]{Conjecture}
\newtheorem{corollary}[theorem]{Corollary}
\newtheorem{criterion}[theorem]{Criterion}
\newtheorem{definition}[theorem]{Definition}
\newtheorem{derivation}{Derivation} % Number derivations on their own
\newtheorem{example}[theorem]{Example}
\newtheorem{exercise}[theorem]{Exercise}
\newtheorem{lemma}[theorem]{Lemma}
\newtheorem{notation}[theorem]{Notation}
\newtheorem{problem}[theorem]{Problem}
\newtheorem{proposition}{Proposition} % Number propositions on their own
\newtheorem{remark}[theorem]{Remark}
\newtheorem{solution}[theorem]{Solution}
\newtheorem{summary}[theorem]{Summary}
%\numberwithin{equation}{section}
\bibliographystyle{aer}
\newcommand\ve{\varepsilon}
\newcommand\boldline{\arrayrulewidth{1pt}\hline}


\begin{document}

\begin{flushleft}
  \textbf{\large{Problem Set \#[1]}} \\
  MACS 30100, Dr. Evans \\
  Soo Wan Kim
\end{flushleft}

\noindent\textbf{Problem 1} \\
\textbf {Part (a)-(b).} \\

\noindent

In "Great Expectations: A Field Experiment to Improve Accountability in Mali," Jessica Gottlieb examines the effects of increasing information about local governments on constituent expectations of local government behavior. The data were drawn from a field experiment where Malian communes were randomly assigned to receive a civics course, and later surveyed to measure civic knowledge, knowledge of local government capacity, and expectations about local government performance. There were two treatments: in Treatment 1 communes were given information about the capacities and responsibilities of the local government, and in Treatment 2 communes received the same information but with additional information on how the local government performed relative to others. \\

\noindent
Citation: \\
\noindent
Gottlieb, Jessica. 2016. "Greater Expectations: A Field Experiment to Improve
\indent\indent
Accountability in Mali." \textit{American Journal Of Political Science} 60, no. 1: 
\indent\indent
143-157. \textit{Business Source Complete}, EBSCOhost (accessed January 7, 2017).\\

\noindent\textbf {Part (c).} \\
for individual i in village v in commune c:
\[y_{ivc} = \beta_0 + \beta_1T1_c + \beta_2T2_c + W'_c\Gamma + Z'_e\Lambda + \alpha_c + \gamma_{vc} + \varepsilon_{ivc} \]
\noindent
where \\
\indent
\textbf{y} is survey outcome \\
\indent
\textbf{T1} and \textbf{T2} indicate whether the commune received Treatment 1 or 2 \\
\indent
\textbf{W} is a fixed effect for block \\
\indent
\textbf{\(Z_e \)} are enumerator (survey administrator) fixed effects \\
\indent
\textbf{\(\alpha\)} are random effects for commune \\
\indent
\textbf{\(\gamma\)} are random effects for village \\

\noindent\textbf {Part (d).} \\
exogenous: T1, T2, W, Z, \(\Gamma \), \(\Lambda \) \\
endogenous: y \\

\noindent
\textbf {Part (e).} \\
The model is static, linear, and stochastic. \\

\noindent
\textbf {Part (f).} \\
gender, age group, education level \\

\noindent\textbf{Problem 2} \\
\textbf {Part (a)-(c).} \\
\noindent
for artist i: 
\[y_i = \beta_0 + \beta_1gender_i + \beta_2education_i + \beta_3hiphop_i + \beta_4country_i + \beta_5pop_i + \beta_6rock_i \]
\[+ \beta_7drugs_i + \beta_8share_i + \beta_9share\_consistency_i + \beta_{10}born \varepsilon_i \]
\\
\noindent
where \\
\indent
\textbf{y} = predicted lifespan in years \\
\indent
\textbf{gender} = 0 if male and 1 if female \\
\indent
\textbf{education} = number of years of formal education completed \\
\indent
\textbf{hiphop} = 1 if hip hop is one of artist's main genres, 0 otherwise\\
\indent
\textbf{country} = 1 if country is one of artist's main genres, 0 otherwise\\
\indent\indent
and so on for other genres (more genres can be added after data exploration) \\
\indent
\textbf{drugs} = 1 if artist is known to have or had an alcohol or hard drug addiction, \\
\indent\indent
0 otherwise \\
\indent
\textbf{share} = total album and single sales for artist over career divided by \\
\indent\indent
sum of sales in the sample \\
\indent
\textbf{share\_consistency} measures fluctation in sales over the artist's career \\
\indent\indent
after the initial hit and is calculated by summing the range of the artist's\\
\indent\indent
share of total sales over  his or her career with the variance: 
\[share\_consistency = share_{peak} - share_{trough} + variance(share) \]
\indent
\textbf{born} = year of birth \\

\noindent
\textbf {Part (d)-(e).}\\
\noindent
\textbf{Demographic factors}: Women generally live longer than men, and more educated individuals should make better-informed decisions that affect their health and lifespan. \\
\textbf{Genre}: Genre is highly likely to influence artists' lifestyle choices. For example, some rock genres and rap are associated with "fast living" - partying, drugs, motorcycles, promiscuity, etc. which is likely to reduce a person's lifespan. Other genres, such as country and gospel, are not. The effects of genre will probably be a combination of various factors such as peer pressure, marketing tactics, and innate personality traits, which are difficult to separate and measure on their own. \\
\textbf{Drugs}: When popular artists die young, hard drug and alcohol abuse appears to frequently be a factor. Drug and alcohol use may be influenced by genre but I expect that having a certain amount of income and the pressures of stardom/art will lead to temptation for most popular musicians eventually.\\
\textbf{Career success}: Different levels of success (which I measure by music sales) will affect artists in different ways. For example, being a minor star as opposed to a major star may lead to lower self-esteem and more self-destructive behavior. Dramatic fluctuations in sales over time may also lead to anxiety and financial insecurity. \\
\textbf{Year of birth} matters because different generations have different life expectancies. Artists born later should live longer on average.\\
\\
Because I employ a linear regression model, I tried to leave out factors that covary with factors in the model. For example, smoking raises health risks but is more prevalent among older generations (covaries with birth year), and race may influence the outcome but tends to overlap with genre (e.g. the most popular rap artists tend to be black). I would expect country effects to be insignificant because most of the artists would come from the US or other Western countries with similar standards of living. I also leave out factors that are tricky to measure such as family health history, mental health history, and family socioeconomic background.\\

\noindent
\textbf {Part (f).}
To test my model, I would first need a dataset with the names and basic demographic information of popular artists. Popularity would be based on international album sales or a simillar criterion and I would set the cut-off at the top 100 artists or acts. I could then incorporate relevant information from preexisting datasets (e.g. ``List of deaths in rock and roll" page on Wikipedia) or scrape information from online biographies and sales records. The measurement/coding criteria will be as described in Part (a)-(c). I will test the regression model in R.



\end{document}
